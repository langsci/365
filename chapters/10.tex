\chapter{Conclusion}\label{chapter10}

This book set out to investigate the production, perception, and comprehension of subphonemic detail. To operationalise the investigation, word-final /s/ in English was used in real word and pseudoword target items for a production task, an implementation of linear discriminative learning, a same-different task, and two number-decision tasks.

The first general aim of the present book was to examine whether durational differences in morphologically different types of word-final /s/ – non-morphemic, plural, \textit{is}-, and \textit{has}-clitic /s/ – can be found and how such differences can be accounted for. While previous studies reported such differences, the nature of these differences deviated between previous corpus studies and previous experimental studies. The results obtained in the production task presented in this book are in line with the findings of previous corpus studies. That is, non-morphemic /s/ is longest in duration, clitic /s/ is shortest in duration, and plural /s/ duration is in between non-morphemic /s/ and clitic /s/ durations. The two clitics under investigation were found not to be significantly different in terms of their durations. Turning to the results of the LDL implementation, it seems that the durational differences are connected to a word’s resonance with the lexicon in that its semantic activation diversity and its phonological certainty are predictors of its word-final /s/ duration.

The second general aim of this book was to investigate how small a durational difference in word-final /s/ is perceptible. Using a same-different task, it was found that listeners showed a higher sensitivity for a durational difference of 35 ms as compared to smaller durational differences. This finding is more or less in line with previous work in that the just-noticeable durational difference should be at about 25 ms, but higher for word-final fricatives as is the case for word-final /s/.

The third general aim of this book was to find out whether subphonemic durational differences significantly influence comprehension. To investigate this issue, two number-decision tasks in a mouse-tracking paradigm were used. One task made use of isolated real words with either durationally matched or mismatched non-morphemic and plural /s/ duration, while the other task used pseudowords embedded within sentences with either durationally matched or mismatched plural, \textit{is}-, and \textit{has}-clitic /s/ duration as target items. It was found that reaction times are not influenced by the mismatch of durational information. However, both comprehension studies found a significant difference between mouse-tracks of trials of matched versus trials of mismatched durational information. Thus, the process of comprehension itself apparently is influenced by subphonemic detail, while the duration of the process of comprehension is not.

The investigation of the general aims revealed that a discernible number of extant models of speech production, perception, and comprehension cannot account for the present findings. Subphonemic durational differences are not predicted at all, or their directions are either unpredicted or said to be the opposite of what was found. The perception of subphonemic durational detail is ruled out completely, and an influence on comprehension is thus not considered. In light of the findings presented in this book, then, such models need to be revised. Yet, some promising, especially computational, approaches already exist. Future implementations of such accounts will show whether and how such approaches can be used to explain the intricacy of language structure.
The complexities of speech production, perception, and comprehension remain enormous. The present book may have shed light on only a few of many issues: the production, perception, and comprehension of subphonemic detail. It was demonstrated by the findings of this book that various theoretical approaches to the production, perception, and comprehension of language and its fine-grained phonetic detail are in need of revision. 
