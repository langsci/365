\chapter{General discussion}\label{chapter09}

In this book, I set out to establish substantiated knowledge on subphonemic detail and its role in production, perception, and comprehension. To achieve this goal, I used real words and pseudowords as items where applicable and I conducted thorough statistical analyses using novel statistical techniques where appropriate. To investigate the production, perception, and comprehension of subphonemic detail I made use of word-final /s/ in English as it is not only found as non-morphemic segment, but also has numerous morphological functions: plural, genitive, genitive plural, third-person singular, as well as the clitics of \textit{is}, \textit{has}, and \textit{us} (as in \textit{let’s}). Using a subset of these different types of /s/ – non-morphemic, plural, \textit{is}-, and \textit{has}-clitic /s/ – I conducted five studies. The aims of these studies were to determine whether such different types of /s/ show differences in their acoustic duration in production (Chapter \ref{chapter04}), to gain further insight into how such durational differences come to be (Chapter \ref{chapter05}), to learn whether durational differences in word-final /s/ are perceptible (Chapter \ref{chapter06}), and to examine if durational differences in word-final /s/ are made use of in comprehension (Chapters \ref{chapter07} and \ref{chapter08}). All investigations were of an explorative nature, addressing hypotheses derived from relevant theories to provide elaborate discussions of the pertinent findings. In the following, the respective hypotheses are repeated and then discussed based on the findings of the individual studies. Finally, all results are brought together to draw an overall picture of the production, perception, and comprehension of subphonemic detail in word-final /s/.

The production study presented in Chapter \ref{chapter04} of this book investigated whether there are durational differences in the acoustics of non-morphemic, plural, \textit{is}-, and \textit{has}-clitic word-final /s/. Using pseudowords as items in a highly controlled production task, it was made sure that effects of lexical frequency, predictability, and storage did not confound the results. It was shown that non-morphemic /s/ was longest, plural /s/ was shorter, and clitic /s/ was shortest. While these differences were found to be significant, the difference between the \textit{is}- and the \textit{has}-clitic was not. The following hypotheses were investigated:

\begin{description}
\item\textsc{H prod\textsubscript{1}}: \textit{Feed-Forward Hypothesis} \newline
There is no durational difference between word-final non-morphemic /s/, plural /s/, and auxiliary clitic /s/.

\item\textsc{H prod\textsubscript{2}}: \textit{Prosodic Hypothesis} \newline
There are durational differences between different types of word-final /s/: 
non-morphemic /s/ is shorter than plural /s/, plural /s/ is shorter than auxiliary clitic /s/.

\item\textsc{H prod\textsubscript{3}}: \textit{Emergence Hypothesis} \newline
There are durational differences between different types of word-final /s/ (non-morphemic, plural, and auxiliary clitic).
\end{description}

\textsc{H prod\textsubscript{1}}, the \textit{Feed-Forward Hypothesis}, is rejected as it predicted no durational differences between different types of word-final /s/. If standard feed-forward models of speech production underlying this hypothesis were refined in such a way that post-lexical processes can arise from certain kinds of lexical information, only then the present findings could be accounted for. \textsc{H prod\textsubscript{2}}, the \textit{Prosodic Hypothesis}, is rejected as it predicted the opposite direction for durational differences, with non-morphemic /s/ being shortest and clitic /s/ being longest in duration. This pattern is clearly not compatible with the present results. The theories underlying \textsc{H prod\textsubscript{3}}, the \textit{Emergence Hypothesis}, can potentially account for the present findings. The fact that durational differences were found indicates that such differences might emerge through the mechanisms introduced by the theories underlying this hypothesis. However, claiming that the hypothesis is therefore confirmed would be a fallacy: Only an implementation of one of such underlying theories can show whether the particular theory and its mechanisms can account for the durational differences found in the present production study.

Hence, an implementation of one of the underlying theories, linear discriminative learning, was used to further investigate the hypothesis. This LDL implementation and its analysis were presented in Chapter \ref{chapter05} of this book. Using the non-morphemic and plural /s/ durational data elicited in the production study, the analyses of the LDL implementation resulted in three main findings. First, measures derived from an LDL network trained on real words and pseudowords are predictive of word-final /s/ duration in pseudowords. Such measures are indeed just as predictive of /s/ durations as are more traditional variables. Second, even though such LDL measures show about the same level of predictivity, the effect of the type of /s/ as a variable is not fully captured by them. That is, the type of the word-final /s/ remained a significant predictor when introduced among measures derived from the LDL network. This indicates that there is more to the type of /s/ than the variables used in the present implementation. Third, even though the type of /s/ is not fully captured by the LDL measures, especially one of these measures, the correlation with the semantic nearest neighbour, showed a high correlation with the type of /s/. Hence, intricate semantic properties of the types of /s/ under investigation are indeed captured by measures derived from the LDL network. Coming back to the hypothesis at hand, \textsc{H prod\textsubscript{3}}, the \textit{Emergence Hypothesis}, it is found that it can be confirmed in regard to one of its underlying theories, linear discriminative learning.

Taking the results on the production of word-final /s/ as a starting point, the perception study presented in Chapter \ref{chapter06} asked whether such durational differences are perceptible. For this, a same-different task with real words and pseudowords as items was conducted. For each item, a version with the pertinent prototypical duration of non-morphemic or plural /s/ was created. Then, four further versions were constructed with their word-final /s/ either being incrementally shortened (mono-morphemic items) or lengthened (plural items) by 10 ms, 20 ms, 35 ms, and 75 ms. The results indicate that, on average, listener sensitivity is rather low for durational differences of 10 ms and 20 ms, and slightly but significantly higher for a durational difference of 35 ms. For the 75 ms durational difference, a significantly improved sensitivity was found. The following hypotheses were investigated:

\begin{description}
\item\textsc{H perc\textsubscript{1}}: \textit{Abstractionist Hypothesis} \newline
Listeners are not sensitive to subphonemic durational differences between different types of word-final /s/.

\item\textsc{H perc\textsubscript{2}}: \textit{Phonetic Detail Hypothesis} \newline
Listeners are sensitive to subphonemic durational differences between different types of word-final /s/.
\end{description}

\textsc{H perc\textsubscript{1}}, the \textit{Abstractionist Hypothesis}, is rejected. The hypothesis was built on theories which assume that subphonemic durational differences are not perceptible. Due to the strictly phonological nature of perception found in such theories, these and the present findings are fully incompatible. \textsc{H perc\textsubscript{2}}, the \textit{Phonetic Detail Hypothesis}, can be confirmed under two premises. First, only an implementation of the models underlying the hypothesis can sufficiently confirm whether a particular model’s mechanisms can account for the present findings. Second, listeners showed sensitivity to subphonemic durational differences. However, major increases in sensitivity and overall high levels of sensitivity were only found for the biggest durational difference of 75 ms – a difference that is not found in studies on the durational differences between different types of /s/. Thus, according to the present findings, not all durational differences between different types of /s/ found in studies on their acoustic duration are assumed to be well perceptible. 

Importantly, there most likely is an issue of methodology at hand here. Same-different tasks such as the one used in the present perception study are metalinguistic tasks. Hence, certain properties of language are the main focus for parti-cipants of such tasks instead of language or language use itself. Thus, participants encountered a task they are not familiar with and that extends beyond their day-to-day usage of language: differentiating isolated words by the duration of their word-final /s/. It might thus very well be the case that a same-different task is not the most appropriate experimental setup to investigate the perceptibility of subphonemic durational differences.

A type of task that focuses more narrowly on language use itself was used in the two comprehension tasks presented in Chapters \ref{chapter07} and \ref{chapter08}. In number-decision tasks, participants were asked to decide whether an isolated word (Chapter \ref{chapter07}) or the agent in a sentence (Chapter \ref{chapter08}) was singular or plural. In the case of isolated words, words with non-morphemic and plural /s/ were used as target items. In the case of agents in a sentence, pseudowords with plural, \textit{is}-, and \textit{has}-clitic /s/ were used as target items. In both experiments, /s/ durations were either matched with their context, e.g. a plural word had an /s/ with a typical plural /s/ duration, or /s/ durations were mismatched with their context, e.g. a plural word had an /s/ with a typical non-morphemic or clitic /s/ duration. It was found that reaction time was not influenced by the durational mismatch of word-final /s/ and (pseudo-)base. Mouse-tracks, however, showed a significant effect of mismatched durations in that they followed significantly different paths as compared to the mouse-tracks of matched items. In both comprehension studies, the following hypothesis was investigated:

\begin{description}

\item\textsc{H comp}: \textit{Mismatch Hypothesis} \newline
If listeners make use of subphonemic durational differences in the comprehension of different types of word-final /s/, then a mismatch of subphonemic detail and intended meaning leads to\newline
a) slowed down comprehension processes.\newline
b) deviated mouse trajectories.

\end{description}

As no differences in reaction times were found in Chapter \ref{chapter07}, part a) of \textsc{H comp} cannot be confirmed. That is, the overall time to react to an audio stimulus with a mismatched /s/ duration is just as long as the time to react to an audio stimulus with a matched /s/ duration. Part b) of \textsc{H comp}, however, is confirmed by the findings, as mouse-tracks of both conditions, matched and mismatched, significantly differed, i.e. the mouse-tracks of the mismatched stimuli trials deviated from the mouse-tracks of the matched stimuli trials. While the patterning of deviation as such is not straightforwardly explainable, especially taking into account the results of Chapter \ref{chapter08}, an influence of mismatched subphonemic durational differences was found, nonetheless.

How do the findings of the individual studies relate to the overarching goal of this book to draw a more detailed, intricate, and exhaustive picture of the production, perception, and comprehension of subphonemic detail? For production, it was found that different types of word-final /s/ are indeed different in terms of their acoustic duration. The nature of these differences is in line with previous corpus studies, but not with previous experimental studies. Analysing the durational differences not only by means of traditional variables but also by using measures derived from an LDL implementation, it was shown that such measures are predictive of word-final /s/ durations. Thus, the origin of durational differences in word-final /s/ can most likely be explained by the resonance of words with the lexicon. Taking into account the highly controlled methodology of the production study, its results, the measures derived from the LDL implementation, and the analyses of these measures, the first general aim of this book can be addressed: Subphonemic durational differences between different types of word-final /s/ exist. A potential explanation for the contradictory nature of previous results lies within the applied methodology and statistical analyses used in previous experimental studies. While these previous studies used homophonous real words as target items, I used pseudowords instead, avoiding the potential issues and uncertainties regarding the representation of homophones within the mental lexicon. Making use of an LDL implementation, pseudowords were shown not to be semantically empty, but to resonate with the lexicon. The measures derived from the LDL implementation, then, allowed for further insight into the origin of such durational differences. That is, higher degrees of semantic activation diversity and higher levels of phonological certainty come with shorter /s/ durations.

For perception, it was found that listeners showed higher sensitivity for durational differences of 35 ms and 75 ms as compared to the smaller differences of 10 ms and 20 ms. The results indicate that durational differences of 35 ms are somewhat perceptible, while durational differences of 75 ms show a further increased level of perceptibility. These results are more or less in line with the findings by \citet{Klatt1975} in that these authors claimed 25 ms to be the just-noticeable durational difference to a segment. As \citet{Klatt1975} also noted that durational differences in word-final fricatives are less well perceptible, the increase in sensitivity and thus perceptibility found for 35 ms is close to their 25 ms, but the threshold is most likely higher due to /s/ being word-final and a fricative. Regarding the second general aim of this book, then, one can conclude that the durational difference to a single segment to be perceptible should be at least of 35 ms if it is a fricative in word-final position. Note, however, the aforementioned issue of the metalinguistic nature of the same-different task on why the overall sensitivity was found to be rather low.

For comprehension, it was found that subphonemic durational differences indeed influence comprehension. Using target items with matched and mismatched durational /s/ information, it was found that mouse-trajectories for matched versus mismatched items were significantly different across all types of /s/ under investigation. This finding suggests that durational differences are used in comprehension, and should thus also be perceptible even though the overall low sensitivity values obtained in the perception task might suggest otherwise. Reaction times, on the other hand, did not significantly differ between matched versus mismatched item trials. However, reaction times only represent a single data point per trial while mouse-tracks give insight into the decision-making process during comprehension. Regarding the third general aim of this book, then, one may conclude that comprehension is influenced by subphonemic durational differences. More precisely, while the time between perception and the outcome of comprehension is not significantly influenced, the comprehension process between the input of an audio stimulus and the outcome of comprehension appears to be significantly affected.

So what does the overall picture of the production, perception, and comprehension of subphonemic detail look like? Subphonemic detail is influenced by morphological make-up as different types of word-final /s/ show differences in their acoustic durations and is perceptible if durational differences are above a certain threshold. Subphonemic detail influences and is made use of in the process of comprehension. As was demonstrated, these overall results ultimately call for revisions of models of speech production, perception, and comprehension which do not incorporate subphonemic detail in their pertinent representations and processes.  
