\chapter{Subphonemic differences in phonologically identical elements}\label{chapter02}

Research of the last decades has repeatedly shown that subphonemic differences are found in the production of phonologically identical elements (\cite{Walsh1983, Hsieh1999, Cho2001, Jurafsky2002, Lavoie2002, Sugahara2004, Sugahara2009, Kemps2005a, Kemps2005b, Gahl2008, Drager2011, Smith2012, Zimmermann2016, BenHedia2017, Plag2017, Seyfarth2017, Lohmann2018, BenHedia2019, Tomaschek2019, Plag2020}) and that such differences can be perceived as well as be used in comprehension (e.g. \cite{Klatt1975, Warner2004, Kemps2005a, Kemps2005b, Shatzman2006}). It is on these findings that the research presented in this book is grounded. Instead of following a specific theory that is to be confirmed, the studies this book reports on are of an explorative nature. The research questions addressed in the individual studies are met with all relevant theories at hand to provide elaborate discussions of the respective findings. The exploratory nature of this research is typical of research spanning multiple areas of a discipline. In the present case, findings, concepts, and approaches of morphology, phonology, phonetics, computational linguistics, and psycholinguistics are combined. While this at times may prove difficult, it also enriches the knowledge gain of the field by combining theoretical accounts spanning different subdisciplines. The overall aim of the approach in this book follows its overarching aim of establishing substantiated knowledge on subphonemic detail and its role in production, perception, and comprehension. 

Previous findings as well as the main theoretical accounts concerning the production, perception, and comprehension of subphonemic detail are introduced in the following sections, Section \ref{section02_1} and Section \ref{section02_2}, respectively. In both sections, I will first review relevant previous empirical findings before I then introduce pertinent theoretical approaches and models. Taking the theoretical accounts as motivation, I will derive the hypotheses to be explored in the studies of this book. Finally, I will sum up the hypotheses for a concise overview in Section \ref{section02_3}.

\section{Production}\label{section02_1}

The evidence for the presence of morphological information at the subphonemic level emerges mainly from the study of homophonous lexemes, stems, and affixes.\protect\footnote{An earlier version of this section has been published in \protect\cite{Schmitz2021a}.} For homophonous lexemes, \citet{Gahl2008} and \citet{Lohmann2018} investigated acoustic realisations of seemingly homophonous word pairs such as \textit{time} and \textit{thyme} and found the more frequent member of each pair to be of shorter acoustic duration. Further evidence for differing acoustic realisations of supposedly homophonous lexemes was found by \citet{Drager2011}. Drager compared realisations of \textit{like} as adverb, verb, discourse particle, and as part of the quotative \textit{be like}. Differences surfaced in several phonetic parameters. Similar effects were found for function words such as \textit{four} and \textit{for} and different uses of words such as \textit{to}, which were investigated by \citet{Lavoie2002} and \citet{Jurafsky2002}. Such fine realisational differences indicate that at the phonetic level, two or more phonologically homophonous lemmas may differ in their realisation.

Similarly, evidence shows that seemingly homophonous elements below the word level have different phonetic realisations. \citet{Kemps2005a} and \citet{Kemps2005b} found that in Dutch and English segmentally identical free and bound variants of a base (e.g. \textit{help} without a suffix versus \textit{help} in \textit{helper}) differ acoustically. \citet{Sugahara2004,Sugahara2009} found phonetic differences between the final segments of a mono-morphemic stem as compared to the final segments of the same stem if followed by a suffix, e.g. in \textit{mist rain} versus \textit{missed rain}. The stem had slightly longer rhymes if followed by certain suffixes. \citet{Seyfarth2017} found that for words ending in fricatives, the durations of a word’s morphological relatives influence the realisation of that word. In their study, stems of multi-morphemic words showed longer durations than similar strings of segments in homophonous mono-morphemic words (e.g. \textit{free} in \textit{frees} versus \textit{freeze}). They concluded that the durational targets of the multi-morphemic word’s relatives influence the word’s duration to such an extent that a durational difference between the respective multi-morphemic word and its homophonous mono-morphemic counterpart arise. A similar effect of morphological relations influencing duration was found for plurals and their bare stems in a corpus-based study by \citet{Engemann2021}.

For prefixes, \citet{Smith2012} found systematic realisational differences for \textit{dis}- and \textit{mis}- between prefixed and so-called pseudo-prefixed words (e.g. \textit{discolour} versus \textit{discover}). Prefixed words showed longer durations and longer voice onset times, among other things. \citet{BenHedia2017} and \citet{BenHedia2019} showed that the more segmentable a prefix, the longer the duration of its nasal.

On the articulatory level, \citet{Cho2001} found evidence for the variability of intergestural timing between identical strings in mono- versus multi-morphemic contexts. In their electropalatographic study, Cho showed that the timing of the gestures for [ti] and [ni] in Korean show more variation when the sequence is mono-morphemic (/mati/ ‘knot’ and /pani/ ‘name’) as compared to the timing of the same gestures in multi-morphemic sequences (/mat-i/ ‘the oldest’ and /pan-i/ ‘class-Nom’), thus indicating that morphological structure is reflected in articulatory gestures, which in turn may lead to correlates in the acoustic signal. Hence, morphology is reflected in the phonetic realisation of otherwise identical strings of segments.

In sum, it seems that there is vast evidence for seemingly homophonous elements, that is, lexemes, bases, and affixes, to differ on the level of speech production. Differences on the level of segments have been reported as well. Previous corpus studies on word-final /s/ in English found realisational differences between non-morphemic, suffix, and clitic variants. \citet{Zimmermann2016} on New Zealand English (data from QuakeBox corpus; \cite{Walsh2013}) and \citet{Plag2017} as well as \citet{Tomaschek2019} on North American English (data from Buckeye Corpus of Conversational Speech; \cite{Pitt2007}) found that non-morphemic /s/ showed longer durations than suffix and clitic /s/. In turn, suffix /s/ also showed longer durations than clitic /s/. While these results draw a clear picture of /s/ duration across morphological categories (including the non-morphemic /s/), they are subject to unbalanced data sets due to the nature of corpora. That is, corpus data may contain a huge number of confounding and moderator variables that experimental data can control for (e.g. \cite{Gries2015}).

Previous experimental studies, however, have reported less consistent results and show some problematic methods and analyses. \citet{Walsh1983} carried out a production experiment with three homophonous word pairs (e.g. \textit{Rex} and \textit{wrecks}). They measured the duration of the word-final /s/ in both the mono- and the multi-morphemic word of each pair in three different conditions. Each word was produced by eight to ten participants. Condition I consisted of an unambiguous context; condition II consisted of a semantically neutral context; condition III consisted of a semantically anomalous context.  While in two of these conditions there was a small difference of 9 ms in the means of the different types of /s/, there was none in the third condition. Still, the authors concluded that ``speakers of English systematically lengthen morphemic /s/" (\citet[204]{Walsh1983}. However, their analysed data set was small (110 observations), included a mixture of common and proper nouns, and no phonetic covariates were integrated in their analysis. Further, instead of applying appropriate inferential statistical methods (e.g. t-tests or more advanced methods), the mean durations of the types of /s/ under investigation were compared impressionistically. Therefore, there are several reasons to be sceptical of their results.

In another study, \citet{Hsieh1999} measured /s/ duration in child-directed speech in data originally elicited for another study (on vowel durations in function words, see \cite{Swanson1994}). The authors found plural /s/ to be longer than third-person singular /s/. However, as the study originally was not designed for this endeavour, half of all plural items occurred sentence-finally, while almost all third-person singular items occurred sentence-medially. The durational difference found between the suffixes may hence have been due to effects of phrase-final lengthening (e.g. \cite{Klatt1976,Wightman1992}) rather than to inherent phonetic differences due to morphological categories. 

In a more recent study, \citet{Seyfarth2017} conducted a production experiment to collect data on non-morphemic, plural, and third-person singular /s/ and /z/ durations. They found the non-morphemic variant to be shorter than the morphemic instances. However, they did not find differences between the voiced and the voiceless allomorphs during their analysis. This may be a worrisome result, especially considering the small number of items with voiceless allomorphs (n = 6) as compared to the high number of items with voiced allomorphs (n = 20) in their data. 

Recently, \citet{Plag2020} found plural and genitive plural /s/ to be of different durations. In their study, the genitive plural suffix showed significantly longer durations as compared to the plural suffix. An overview of the durational differences found in the aforementioned experimental studies is given in \ref{tab:2.1}.

\begin{table}\fontsize{10}{11}
\caption{Overview of durational differences of word-final /s/ found in previous studies.}
\label{tab:2.1}
\begin{tabular}{ll}
\lsptoprule
Study                                & Findings                                         \\
\midrule
\cite{Zimmermann2016, Plag2017}; & \multirow{2}{*}{non-morphemic >>
plural >> clitics}  \\
\cite{Tomaschek2019}               &                                                  \\
\cite{Walsh1983}                  & plural >> non-morphemic                            \\
\cite{Hsieh1999}                   & plural >> third-person singular                    \\
\cite{Seyfarth2017}                & plural >> non-morphemic                            \\
\cite{Plag2020}                    & genitive plural >> plural                 \\
\lspbottomrule
\end{tabular}
\end{table}

In sum, there is evidence that there may be durational differences between different types of /s/. However, while results of corpus studies are in line with each other, they might be flawed due to imbalanced data sets. Previous experimental studies, on the other hand, have often relied on small data sets and lacked phonetic covariates, appropriate statistical methods, or a proper distinction of voiced and voiceless segments. Another crucial difference between corpus and experimental studies is the use of homophones. While all previous experimental studies restricted their data to homophone pairs, corpus studies take into consideration all words. The limitation to homophones and the resulting competition between their representations might be a problem in itself, as it appears to be unclear how members of homophone pairs are stored and connected to their respective frequencies. In all cases, previous results were subject to potentially confounding effects of the lexical properties (e.g. effects of frequency, e.g. \cite{Gahl2008,Lohmann2018}; effects of storage, e.g. \cite{Caselli2016}) and contextual effects (e.g. phrase final lengthening, e.g. \cite{Klatt1976,Wightman1992}) of the items under investigation. Also, so far, no experimental study included clitics in their analysis, whereas corpus studies have suggested that clitics show different durations than suffixes.

A study is therefore called for that investigates the durational nature of different types of word-final /s/ in English, preferably an experimental study with carefully controlled data avoiding potentially confounding effects. This book presents such a study investigating word-final /s/ in English by means of a pseudoword production task. In this task, three types of word-final /s/ were elicited: mono-morphemic, plural, and clitic /s/ (with the auxiliaries \textit{is} and \textit{has}). It will address some of the issues of previous studies. More precisely, the use of pseudowords prevents potential lexical effects to confound findings (see Section \ref{section03_1_1}), while the highly controlled task evades the influence of contextual effects. Even though the data will also contain homophones to a certain extent, the individual members do not have lexical representations. That is, one can rule out effects of competition between homophonous lexical entries due to their similar representations. In addition, the use of pseudowords eliminates potential differences in duration due to differences in frequency between the homophones.

Let us now turn to the question of how morpho-phonetic effects can be explained at the theoretical level. Existing theories make different predictions concerning the possible presence of durational differences between different types of /s/. I will discuss four approaches here: feed-forward models of phonology-morphology interaction, Prosodic Phonology, exemplar theory, and discriminative learning. 

In standard feed-forward formal theories of morphology-phonology interaction, all types of /s/, be they morphemic or non-morphemic, are treated in a similar way (e.g. \cite{Chomsky1968,Kiparsky1982}). In the case of morphological word-final /s/, a process called ‘bracket erasure’ is said to remove all morphological information from a pertinent word form once retrieved from the lexicon during the stage of ``lexical phonology" and leaves speech production without an insight into the morphological makeup at the stage of ``post-lexical phonology". After retrieval, there is no informational difference between word-final morphemic and non-morphemic types of /s/. Thus, there is nothing in such a system that could account for realisational differences, e.g. different durations, between phonologically identical suffixes and non-morphemic segments. The realisation of clitics is a post-lexical process to begin with and thus outside the scope of any prediction by this theory.

In the framework of Prosodic Phonology, there is a complex mapping of morphological structure onto prosodic structure (\cite{Booij1983,Nespor2007}). Since prosodic boundaries may correlate with particular phonetic properties, segments at such boundaries may show systematic differences in phonetic implementation (see, for example, \cite{Keating2006}). Phonetic differences between two phonologically homophonous affixes could therefore result from a difference in the prosodic structure that goes with the two affixes. In particular, different types of word-final /s/ can be analysed as having different positions in the hierarchical prosodic configuration. These configurations co-determine the degree of integration of an /s/ to the word it belongs to. These different degrees of integration might then emerge as durational differences between types of /s/ in speech production. 

Applying the approach of \citet{Selkirk1996}, non-morphemic /s/, uncontroversially, is an integral part of the prosodic word, as shown in Panel A of Figure \ref{fig:2.1}. \citet{Goad1998} analyses plural /s/ as an ``internal clitic", which is adjoined to the highest prosodic constituent below the prosodic word, as shown in Panel B. In \citet{Goad2002}, however, plural /s/ is analysed as an ``affixal clitic", like third-person singular /s/ in \citet{Goad2003} and \citet{Goad2019}, as shown in Panel C. The prosodic status of the cliticized auxiliary /s/ is not entirely clear, but presumably, it is best analysed as ``free clitic", as in Panel D.

\begin{figure}
    \caption{Prosodic structure of non-morphemic (A), plural (B, C), and clitic /s/ (D) as given in literature on Prosodic Phonology.}
    \label{fig:2.1}
    \centering
\begin{tabular}{llll}
A               & B                 & C                & D              \\
non-morphemic /s/ & plural /s/          & plural /s/         & clitic /s/       \\
                & `internal clitic' & `affixal clitic' & `free clitic'  \\
                ~ & ~ & ~ & ~ \\
\Tree[.PhPhrase [.Pword [.Syllable \textit{bus} ]]]                & 
\Tree[.PhPhrase [.Pword [.Syllable \textit{cat} ] [.\textit{s} ]]]                &
\Tree[.PhPhrase [.Pword [.Pword [.Syllable \textit{cat} ]] [.\textit{s} ]]]  &                
\Tree[.PhPhrase [.Pword [.Syllable \textit{cat} ] ] [.\textit{s} ] ]
\end{tabular}
\end{figure}

The Prosodic Phonology approach thus posits a structural prosodic difference between non-morphemic /s/, plural /s/, and clitic /s/. This prosodic difference might be mirrored in durational differences. It is, however, not so clear what particular phonetic effects this approach would predict and by which processing mechanism the structural prosodic differences would be translated into different articulations. The most plausible prediction would be that closer integration into the prosodic word would correlate with shorter durations: Non-morphemic /s/ should be shortest, clitic /s/ longest, and plural /s/ in between. From the perspective of phrase-final lengthening (e.g. \cite{Klatt1976, Wightman1992}), one should also expect that clitic /s/ is longest, as it immediately precedes a phrase boundary.

The distinction of lexical and post-lexical processing as introduced by the aforementioned standard feed-forward theories of morphology-phonology interaction is also an integral part of established theories in psycholinguistics. According to models of speech production such as the one proposed by Levelt et al. (\cite{Levelt1999}; see \cite{Roelofs2019} for an update), morphemic /s/ would not differ in realisation from corresponding non-morphemic realisations of /s/. In such models, meanings are stored in the mental lexicon with their forms being represented phonologically. A module called ``articulator" uses these phonological forms for speech production, hence, has no information on the lexical origin of particular segments. As a consequence, in this architecture, no systematic differences between different types of /s/ should emerge.

In contrast, exemplar-based models (e.g. \cite{Goldinger1998, Bybee2001, Pierrehumbert2001, Pierrehumbert2002, Gahl2006}) have an architecture that would in principle allow for morpho-phonetic effects. In such models, lexemes are linked to a frequency distribution over their phonetic outcomes as experienced by the individual speaker. These distributions are updated with each new experience: Experienced subtle subphonemic differences then may result in representations mirroring these properties. While such an account may allow for durational differences between different types of word-final /s/ to emerge from stored phonetic representations, it leaves open the question of how such systematic differences between clouds of exemplars would come about in the first place. The downside of this is that it is also unclear in which direction differences between different types of /s/ should play out.

Finally, there is the discriminative learning approach, which is based on simple but powerful principles of discriminative learning theory (\cite{Rescorla1988, Ramscar2007, Ramscar2010}; see, for example, \cite{Baayen2011, Baayen2019} for its application to linguistic problems). According to this theory, learning results from exposure to informative relations among events in the individual’s environment. Individuals use the associations between these events to create cognitive representations of their environment. Most importantly, associations and their resulting representations are updated constantly on the basis of new experiences. Associations are built between features (``cues", e.g. biphones) and classes or categories (``outcomes", e.g. different types of /s/) that co-occur in events in which the learner is predicting the outcomes from the cues (\cite{Tomaschek2019}). The relation between cues and outcomes is modelled mathematically by the so-called Rescorla-Wagner equations (\cite{Rescorla1972, Wagner1972, Rescorla1988}). Following these equations, an association strength or ``weight" increases every time a cue and an outcome co-occur, while it decreases if a cue occurs without the outcome in a learning event. This results in a continuous recalibration of association strengths, which is a crucial part of discriminative learning. 

In recent discriminative learning implementations, the association weights between semantic representations and phonetic representations have been shown to be predictive of phonetic durations (e.g. \cite{Stein2021}). With regard to final /s/, \citet{Tomaschek2019} show that the different durations of final /s/ can be understood as following from the extent to which words’ phonological and collocational properties can discriminate between the inflectional functions expressed by the /s/. The input features (cues) for their discriminative network were the words (``lexomes" as pointers to the meaning of the forms) in a five-word window centred on the /s/-bearing word and the biphones in the phonological forms of these words. These cues are associated with the inflectional functions of the /s/. Two main measurements emerged as significant predictors of /s/ duration. The so-called ``activation" (named ``prior" in \cite{Tomaschek2019}) is a measure of an outcome’s baseline activation, i.e. of how well an outcome is entrenched in the lexicon. The other measure is ``activation diversity", which quantifies the extent to which the cues in the given context also support other targets. The general pattern now is the following: When the uncertainty about the targeted outcome increases, the acoustic duration of /s/ decreases. In other words, stronger support (both from long-term entrenchment and short-term from the context) for a morphological function leads to a longer, i.e. enhanced, acoustic signal. 
In sum, the discriminative approach predicts that differences between different types of /s/ may emerge from the associations of form and meaning that speakers develop as a result of their experience with the pertinent words. But what about pseudowords?  It has recently been shown by \citet{Chuang2021} that these associations also play a role for pseudowords. Pseudowords have no representation in the lexicon, but, as these authors show, pseudowords nevertheless resonate with the lexicon due to their formal similarity with existing words. This resonance even influences subtle phonetic details such as duration. It is, however, yet unclear what kinds of durational differences can be expected between different types of /s/ in pseudowords.

Effects of informativity or predictability (which are also inherently present in discriminative learning approaches) are also to mention, as they may play a role as well (\cite{Seyfarth2014, Priva2015, Zee2021}). Greater predictability of the word in its context has been found to lead to phonetic reduction, for example, to shortening in duration. On the other hand, higher paradigmatic predictability has been shown to correlate with longer duration (``paradigmatic enhancement", e.g. \cite{Kuperman2007, Bell2021}). As these informativity effects are necessarily bound to existing words, an experiment that uses pseudowords cannot straightforwardly test these approaches. 

Based on the different theories laid out above, different hypotheses about durational differences between different types of /s/ in pseudowords can be set up. \textsc{H prod\textsubscript{1}}, the ``Feed-Forward Hypothesis", arises from feed-forward approaches and is in accordance with the prediction that no systematic phonetic differences should be observed between different types of /s/. \textsc{H prod\textsubscript{2}}, the ``Prosodic Hypothesis", is derived from prosodic approaches. According to these approaches, a higher degree of prosodic integration should correlate with shorter durations. Hence, non-morphemic /s/ should be shorter than plural /s/, and plural /s/ should be shorter than clitic /s/. Finally, exemplar-based approaches and discriminative learning approaches both predict the presence of morpho-phonetic effects, but it is unclear how these differences would play out for the three types of /s/ in the present production study. This is encapsulated in \textsc{H prod\textsubscript{3}}, the ``Emergence Hypothesis". 

In summary, the production study presented in Chapter \ref{chapter04} of this book intends to establish whether there are durational differences also with nonce words, and if so, how these differences play out.

\begin{description}
\item\textsc{H prod\textsubscript{1}}: \textsc{Feed-Forward Hypotheses} \newline
There is no durational difference between word-final non-morphemic /s/, plural /s/, and auxiliary clitic /s/.

\item\textsc{H prod\textsubscript{2}}: \textsc{Prosodic Hypotheses} \newline
There are durational differences between different types of word-final /s/: 
non-morphemic /s/ is shorter than plural /s/, plural /s/ is shorter than auxiliary clitic /s/.

\item\textsc{H prod\textsubscript{3}}: \textsc{Emergence Hypotheses} \newline
There are durational differences between different types of word-final /s/ (non-morphemic, plural, and auxiliary clitic).
\end{description}

\section{Perception and comprehension}\label{section02_2}

Findings on subphonemic durational differences give rise to two further questions. First, are listeners able to perceive such subphonemic durational differences between different types of word-final /s/? That is, are listeners not only sensitive to differences between different phonemes (e.g. \cite{Goldstone2010}) but can they pick up on differences between phonologically similar but phonetically different realisations? Second, if subphonemic durational differences are perceptible, are they used in comprehension? That is, does the perception of (un-)expected subphonemic features influence the comprehension process?

On the level of word perception and comprehension, \citet{Shatzman2006} showed that listeners make use of segment durations as a cue for word boundaries. In their study, native speakers of Dutch listened to ambiguous sentences in which plosive-initial words, e.g. \textit{pot} ‘jar’, were preceded by \textit{eens} ‘once’. Additionally, the sentences could also refer to cluster-initial words instead, e.g. \textit{een spot} ‘a spotlight’. The two readings were, among other acoustic features, different in regard to their /s/ durations: Word-initial /s/ was overall longer in duration than word-final /s/ (Δ = 51 ms). The authors found that listeners make use of such different durations for their lexical decision. That is, the durational difference between word-initial and word-final /s/ was perceptible and used for an informed lexical decision, i.e. in word comprehension.

\citet{Warner2004} investigated whether listeners perceive subphonemic differences in Dutch words of identical phonetic but different underlying phonological form, e.g. /met/ ‘measures (sg.)’ and /med/ ‘avoided (sg.)’, where both words phonetically are transcribed as [meit]. Productions of such word pairs showed differences in the subphonemic features between the members of a pair. One of these features was vowel duration, which listeners showed sensitivity to: Listeners were able to perceive subphonemic detail and to use this information in comprehension, even though differences were rather small, e.g. Δ = 3.5 ms for vowel duration. 

\citet{Kemps2005a} and \citet{Kemps2005b} found that listeners in Dutch and English are sensitive to the durational differences between stems in isolation and stems as parts of plural word forms, e.g. \textit{help} without a suffix versus \textit{help} in \textit{helper}. This finding is confirmed by similar results in \citet{Lee2020} and in \citet{Blazej2015}. In their study,  Blazej and Cohen-Goldberg showed that listeners make use of duration as a cue for distinguishing unsuffixed stems from suffixed stems, e.g. \textit{clue} without a suffix versus \textit{clue} in \textit{clueless}. The authors found the influence of duration as a cue to be persistent in isolated and continuous speech, with full, reduced, and removed effects of articulation, and in implicit and explicit tasks. 

Taking into account the aforementioned findings, the question arises what the just-noticeable difference to be perceived is. \citet{Klatt1975} found this difference for a change in duration to a single segment to be 25 ms. That is, below the durational difference of 51 ms found in \citet{Shatzman2006} but well above the durational difference of 3.5 ms given in \citet{Warner2004}. Further, according to Klatt and Cooper’s findings, this just-noticeable difference threshold is influenced by several factors. Most importantly, differences in word-final position and differences in fricatives are less well perceptible. 

In sum, evidence for the perception of subphonemic differences in phonologically similar segments and its effect on comprehension exists. However, such evidence is rather sparse and mainly concerned with lexical decisions or differentiation of unsuffixed and suffixed forms. To date, there is no study which looks into the perception and comprehension of phonologically identical but phonetically and morphologically different segments. Thus, two types of studies are called for. First, a study is needed that investigates whether durational differences found between such segments are perceptible. This is the aim of the same-different task I present in Chapter \ref{chapter06} of this book. Using real words as well as pseudowords, potential lexical effects are taken into account. Second, it needs to be investigated whether subphonemic detail is not only perceptible but also used in comprehension. This is the purpose of the two number-decision mouse-tracking tasks I present in Chapters \ref{chapter07} and \ref{chapter08}. Using isolated real words with non-morphemic and plural /s/ in one of the tasks, and pseudowords with plural and clitic /s/ embedded in real word contexts in the other, a detailed image of whether comprehension is affected by subphonemic durational differences is drawn. That is, evidence for real words as well as for pseudowords and for several types of word-final /s/ will be illustrated.

Let us now turn to the question of how the perception and comprehension of subphonemic detail can be explained at the theoretical level. Existing theories of speech perception and comprehension make different predictions concerning the perception of subphonemic detail and its use in comprehension. I will discuss several groups of approaches here: Theories that make use of abstract representations, theories that rely on sets of features, theories that combine abstract representations and sets of features, and computational models of speech perception and comprehension.

In abstractionist models of speech perception, the phonetics of the incoming speech signal are translated into phonemic representations before the stage of lexical access. That is, the result of perception is of phonological nature and without information on phonetic detail. Well known examples of abstractionist approaches are the TRACE model (e.g. \cite{McClelland1986}), Shortlist (e.g. \cite{Norris1994}) and Shortlist B (\cite{Norris2008}) as well as the speech perception and lexical access model introduced by \citet{Klatt1979}. All of these models have in common that perception of subphonemic detail is either considered to be a peripheral process at the margins of speech perception or that it is not considered at all. Additionally, some abstractionist models (e.g. \cite{Klatt1979}) perform time normalisation. Timing (and with that duration) is only conceived as important if it serves a discriminative role, e.g. in stress placement. As this group of abstractionist models does not integrate subphonemic detail in the process of perception, it cannot account for the perception of subphonemic detail and, consequently, its use in comprehension. If subphonemic detail is not considered for the outcome of the perception process, there is no need to perceive it in the first place. Thus, comprehension has no access to any subphonemic, pre-phonological-representation information.

Approaches that make use of features instead of abstract phonemic representations form another group of speech perception models. One such model is the Fuzzy Logical Model of Speech Perception (\cite{Massaro1987}). It assumes that multiple sources of information influence speech perception, that listeners have continuous information about each source, and that the multiple sources are used together in the most meaningful manner. Sources contribute features of sounds as information, which are then used to build so-called summary descriptions. These, in turn, are the result of the perception process. That is, comprehension does not make use of abstract phonological representations as in abstractionist models but of sets of distinct features. Another model based on features was introduced by \citet{Lahiri1991}. Their approach assumes that there is a single underlying phonological representation per lexical item, which is compatible with all phonologically permissible variants of it in a given context. Entailed in such representations is only marked information, i.e. phonetic features. Concerning the perception of subphonemic durational differences, then, one may regard the two aforementioned models as inconclusive. If subphonemic segmental durational differences are accounted for as a meaningful feature, i.e. if it is assumed to be marked information, perception of durational differences in word-final /s/ can be accounted for. Then, such differences can be used in comprehension. If, however, duration is only considered a feature where it distinguishes between phonemes, then perception of subphonemic durational differences is uncalled for. As a consequence, subphonemic durational differences cannot be used in comprehension.

Exemplar-based models of speech perception (e.g. \cite{Goldinger1996}) also rely on features. They assume that individuals draw on a multitude of exemplars per word form, which are all stored in their mental lexicon. Exemplars contain detailed phonetic information, which gives space to information on subphonemic detail. In this regard, exemplar based models account for the perceptibility of subphonemic durational differences, as such differences are stored in exemplars and made use of in perception and comprehension. 

However, previous research has shown that effects attributed to exemplars are not consistently found (e.g. \cite{Hanique2013Aalders}). Such findings are the motivation for hybrid models. One such hybrid model introduced by \citet{Pierrehumbert2002} assumes abstract generalisations as well as exemplars associated with phonological units, that is phonemes, phoneme sequences, and words. While speech production makes use of both abstract representations and exemplars, comprehension mainly relies on the exemplars. Another hybrid model, Polysp (Polysystemic Speech Perception), has been introduced by \citet{Hawkins2001}. Their model assumes that the analysis of acoustic input does not necessarily rely on its transformation into its linguistic units. Rather, it is situation-dependent whether the abstract phonological form of a word or one of its phonetic variants is accessed for comprehension. As phonetic detail is stored in hybrid models, such models can account for the perception of subphonemic differences and the usage of such differences in comprehension.

The final group of approaches to speech perception and comprehension consists of computational models. One such approach is DIANA, an end-to-end computational model of human word comprehension (\cite{tenBosch2015, tenBosch2021}). The implementation of DIANA supports not only the use of abstract units but also takes exemplars, i.e. phonetically rich information, as input for the modelling of comprehension. Thus, it avoids the assumption of a segmental prelexical layer between acoustic signal and the lexical layer, i.e. perception and comprehension. Similarly, linear discriminative learning (\cite{Baayen2019}; see also Sections \ref{section02_1} and \ref{section03_3}) does not assume a segmental representation layer for acoustic input. Instead, it makes use of Frequency Band Summary Features (FBSFs; \cite{Arnold2017}) as representations. FBSFs consist of detailed information of small time intervals of the signal, containing, for example, information on minimum, maximum, median, initial, and final intensity values. The FBSFs of the complete set of acoustic input then are the result of perception, which is a detailed representation of the perceived phonetic detail. This representation is used in comprehension modelling. In sum, both computational approaches, DIANA and linear discriminative learning, assume detailed phonetic information to be the result of perception, and this information is then used for comprehension. Thus, such models can account for the perception of subphonemic differences and their usage in comprehension. 

Based on the approaches laid out above, two hypotheses about the perception of subphonemic durational differences were formulated. \textsc{H perc\textsubscript{1}}, the ``Abstractionist Hypothesis", arises from models of speech perception that have an abstract phonological representation as outcome of perception. If all fine-grained phonetic detail is lost in perception, one needs not perceive it to begin with. \textsc{H perc\textsubscript{2}}, the ``Phonetic Detail Hypothesis", takes exemplar and hybrid models as well as the aforementioned computational models as a starting point to account for the perception of subphonemic detail. Based on the assumption of storage and usage of detailed phonetic information, subphonemic durational differences can be stored and should thus be perceptible. A hypothesis based on models which make use of features alone is not considered in this book, as predictions on the perceptibility of subphonemic durational differences by such approaches are inconclusive and thus not testable in the current setup.

\begin{description}
\item\textsc{H perc\textsubscript{1}}: \textsc{Abstractionist Hypotheses} \newline
Listeners are not sensitive to subphonemic durational differences between different types of word-final /s/.

\item\textsc{H perc\textsubscript{2}}: \textsc{Phonetic Detail Hypotheses} \newline
Listeners are sensitive to subphonemic durational differences between different types of word-final /s/.
\end{description}

Finally, \textsc{H comp}, the ``Mismatch Hypothesis", emerges as a consequence of the prior two hypotheses. That is, if fine-grained phonetic detail is perceptible, listeners may make use of it in comprehension. Thus, comprehension should be affected if subphonemic detail does not match its intended meaning or context. This influence may be visible in behavioural data, such as reaction times and mouse trajectories. This hypothesis is supported by the exemplar-based, hybrid, and computational approaches.

\begin{description}

\item\textsc{H comp}: \textsc{Mismatch Hypotheses} \newline
If listeners make use of subphonemic durational differences in the comprehension of different types of word-final /s/, then a mismatch of subphonemic detail and intended meaning leads to\newline
a) slowed down comprehension processes.\newline
b) deviated mouse trajectories.

\end{description}

The perception study presented in Chapter \ref{chapter06} aims to establish whether durational differences in word-final /s/ are perceptible. The two comprehension studies of Chapters \ref{chapter07} and \ref{chapter08}, then, investigate whether subphonemic detail is made use of in comprehension.

\section{Summary}\label{section02_3}

To summarise, this book aims at investigating three main areas potentially affected by subphonemic detail: production, perception, and comprehension. Previous findings and relevant theoretical accounts were illustrated in the present chapter. 

The five subsequent chapters will each discuss one study. In Chapter \ref{chapter04}, I will report on the production study that investigates whether durational differences between different types of word-final /s/ are also found in pseudowords. For this study, the following hypotheses are relevant:

\begin{description}
\item\textsc{H prod\textsubscript{1}}: \textsc{Feed-Forward Hypotheses} \newline
There is no durational difference between word-final non-morphemic /s/, plural /s/, and auxiliary clitic /s/.

\item\textsc{H prod\textsubscript{2}}: \textsc{Prosodic Hypotheses} \newline
There are durational differences between different types of word-final /s/: 
non-morphemic /s/ is shorter than plural /s/, plural /s/ is shorter than auxiliary clitic /s/.

\item\textsc{H prod\textsubscript{3}}: \textsc{Emergence Hypotheses} \newline
There are durational differences between different types of word-final /s/ (non-morphemic, plural, and auxiliary clitic).
\end{description}

Chapter \ref{chapter05} will present the implementation of a linear discriminative learning network that was used to analyse the data on non-morphemic and plural /s/ elicited in the aforementioned production study. This study comes without specific hypotheses. Rather, it was used to further investigate H PROD3, that is to explore how the discriminative learning approach might account for durational differences of different types of word-final /s/.

The third study, which constitutes Chapter \ref{chapter06}, investigated the perception of durational differences in word-final /s/. The hypotheses derived for this study are the following:

\begin{description}
\item\textsc{H perc\textsubscript{1}}: \textsc{Abstractionist Hypotheses} \newline
Listeners are not sensitive to subphonemic durational differences between different types of word-final /s/.

\item\textsc{H perc\textsubscript{2}}: \textsc{Phonetic Detail Hypotheses} \newline
Listeners are sensitive to subphonemic durational differences between different types of word-final /s/.
\end{description}

Finally, I will report on the two comprehension studies in Chapters \ref{chapter07} and \ref{chapter08}. The first comprehension study used real words with non-morphemic and plural /s/ in isolation as stimuli, while the second comprehension study used pseudowords with plural and clitic /s/ embedded into real word contexts as stimuli. For both studies, this is the relevant hypothesis:

\begin{description}

\item\textsc{H comp}: \textsc{Mismatch Hypotheses} \newline
If listeners make use of subphonemic durational differences in the comprehension of different types of word-final /s/, then a mismatch of subphonemic detail and intended meaning leads to\newline
a) slowed down comprehension processes.\newline
b) deviated mouse trajectories.

\end{description}

While each study comes with its individual methodological details, they also share some general methodology. In the next chapter, I will outline this general method applied across all studies, including the sets of stimuli and the foundations of the statistical analyses.

